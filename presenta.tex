%        File: presenta.tex
%     Created: Thu Aug 16 04:00 PM 2012 C
% Last Change: Thu Aug 16 04:00 PM 2012 C
%
%      Author: Jorge Villaseñor

\documentclass[hyperref={pdfpagelabels=false},xcolor=pst,pdf,fragile]{beamer}


\providecommand\thispdfpagelabel[1]{}
\usepackage{lmodern}
\usepackage[utf8]{inputenc}
\usepackage{listings}
%\usepackage[spanish]{babel}

%Copenhagen, Warsaw
\usetheme{Boadilla}
\usefonttheme{serif}

\def\Author{Jorge Villaseñor}
\def\Matricula{}
\def\Teacher{}
\def\Title{Open Source Development}



\title{\Title}
\author{\Author\\ \Matricula}
\date{\today}

\begin{document}
\maketitle

\AtBeginSection[]
{
  \begin{frame}
    \frametitle{Outline}
    \tableofcontents[currentsection]
  \end{frame}
}

\AtBeginSubsection[]
{
  \begin{frame}
    \frametitle{Outline}
    \tableofcontents[currentsection,currentsubsection]
  \end{frame}
}

\section{What is Free/Open Source}

\begin{frame}
  \frametitle{General definition of Open Source}

  \begin{itemize}
	\item Access to source code.
	\item Distribution responsabilities.
  \end{itemize}<++>

\end{frame}

\begin{frame}
  \frametitle{Open Source Initiative}

  \begin{itemize}
	\item Founded bye Bruce Perens and Eric Raymond.
  \end{itemize}<++>

\end{frame}

\begin{frame}
  \frametitle{Free Software Fundation}

  \begin{itemize}
	\item Founded by Richard Stallman in 1985.
	\item Promote user freedoms.
	\item Develops the GNU Project.
  \end{itemize}<++>

\end{frame}

\section{Open vs Free Software}

\begin{frame}
  \frametitle{OSI Open Source Definition}

  \begin{itemize}
	\item Based on Debian Free Software Guidelines
	\item Adapted by Bruce Perens (Debian, LSB, Busybox)
  \end{itemize}<++>

  Definition:
  \begin{itemize}
	\item Royality free distribution
	\item Must include Source Code.
	\item Allow derived work.
	\item No discriminnation against persons, groups, fields or endeavor.
	\item License must not restrict other software.
  \end{itemize}

\end{frame}

\begin{frame}
  \frametitle{The Free Software Definition}

  Users have the fredom to:

  \begin{itemize}
	\item Run the program for any purpose (freedom 0)
	\item Study how the program works and change it  so it does you
	  computing as you wiesh (Freedom 1). Access to source code is a
	  precondition on this.
	\item Redistribute copies so you can help your neighbor (freedom 2).
	\item Distribute copies of your modified versions to others (freedom
	  3). By doing this you can give the whole community a chanse to
	  benefit from your changes. Access to source code is a precondition
	  for this.
  \end{itemize}

\end{frame}

\section{History of Free Software}<++>

\begin{frame}
  \frametitle{Before Unix there was caos}

  \begin{block}{OS were written in assembler}
	  \begin{itemize}
		  \item Non portable OS.
		  \item DEC: TOPS-10
		  \item MIT/PDP-10: had ITS (Incompatible Time-sharing System.
	  \end{itemize}
  \end{block}

  \begin{itemize}
	  \item Each community start developing its own OS.
	  \item Some machines started to EOL which lead to drop all specific
		  ASM software.
	  \item ARPANET appears and make possible interconnection cross-geo.
	  \item \alert{Each community shared their Software}
  \end{itemize}

\end{frame}

\begin{frame}
  \frametitle{At the beginning there was Unix}

  There were some important factors leading to Unix

  \begin{itemize}
	  \item Ken Thomson worked on Multics (time-sharing OS)
	  \item Dennis Ritchie invented a portable language C
	  \item Thomson used C to create Unix
  \end{itemize}

  \pause

  \begin{alertblock}{Unix was a portable OS!}
	  Writen in C it can be compiled for PDP-10, PDP-11, x86, etc.
  \end{alertblock}

\end{frame}

\begin{frame}
  \frametitle{Unix got together the hacker community}

  DEC decided to cancel all development on the PDP-10 to focus on PDP-11
  forcing hackers to move to another OS, Unix.

  Pros:
  \begin{itemize}
	  \item Most hackers moved to Unix.
	  \item Software were ported to Unix.
  \end{itemize}

  Cons:
  \begin{itemize}
	  \item Unix was proprietary.
	  \item The future of Unix were determined by a single company risking
		  the community to rewrite once the owner decide to phase out Unix.
	  \item It was impossible to improve the OS.
  \end{itemize}

\end{frame}

\begin{frame}
  \frametitle{The rise of GNU}

  \begin{itemize}
	  \item Richard Stallman founded the GNU project.
	  \item RMS started to write a \alert{free} entire clone of Unix written in C.
	  \item RMS \& GNU got with them the ITS culture to the new Unix one.
	  \item The GNU project started writing the Hurd kernel for the GNU OS.
  \end{itemize}

  \hskip-20pt
  \pause
  \begin{center}
	  Hurd is still in development\ldots
  \end{center}

\end{frame}

\begin{frame}
  \frametitle{Just a kernel is needed}

  \begin{itemize}
	  \item GNU Needed a kernel to get a complete OS.
	  \item Linus started writing an OS kernel as a small project while
		  studying.
  \end{itemize}

\end{frame}

\begin{frame}[fragile]
  \frametitle{What would you like to see most in minix?}

  \begin{lstlisting}
	  Hello everybody out there using minix -
I'm doing a (free) operating system (just a hobby,
won't be big and professional like gnu) for
386(486) AT clones.  This has been brewing
since april, and is starting to get ready.
I'd like any feedback on things people like/dislike
in minix, as my OS resembles it somewhat (same
physical layout of the file-system (due to
practical reasons) among other things).

  \end{lstlisting}

\end{frame}

\begin{frame}[fragile]
  \frametitle{What would you like to see most in minix?}

  \begin{lstlisting}
I've currently ported bash(1.08) and gcc(1.40), and
things seem to work.  This implies that I'll get
something practical within a few months, and I'd like
to know what features most people would want.
Any suggestions are welcome, but I won't promise
I'll implement them :-)

  Linus (torv...@kruuna.helsinki.fi)

PS.  Yes - it's free of any minix code, and it has a
multi-threaded fs.  It is NOT protable (uses 386 task
switching etc), and it probably never will support anything
other than AT-harddisks, as that's all I have :-(. 
  \end{lstlisting}

\end{frame}

\begin{frame}
  \frametitle{Just a kernel is needed}

  \begin{itemize}
	  \item GNU Needed a kernel to get a complete OS.
	  \item Linus started writing an OS kernel as a small project while
		  studying.
  \end{itemize}

  \hskip-20pt

  \begin{itemize}
	  \item This new kernel got a lot of attention from the hacker
		  community. 
		  \pause
	  \item \alert{Linus started to accept collaboration from hackers}
  \end{itemize}
\end{frame}

\section{Why Open Source}

\begin{frame}
  \frametitle{FLOSS represent a lot of advantages}

  \begin{itemize}
	\item Share knowledge.
	\item Colaboration.
	\item Code Review
	\item Allow people to learn
  \end{itemize}

\end{frame}

\begin{frame}
  \frametitle{Exists some common misconceptions}

  \begin{block}{You can not do comercial Open Source}
	\begin{itemize}
	  \item RedHat
	  \item SUSE
	  \item IBM
	  \item Intel
	\end{itemize}
  \end{block}

  \begin{block}{You loose control over your application}
	It depends completly on how you manage it. You must take care of the
	gateways.
  \end{block}

\end{frame}

\section{Cathedral vs Bazaar}

%Unix Philosophy

\begin{frame}
  \frametitle{Bazaar SW Development Rules}

  \begin{itemize}
	\item Every good work of software starts by scratching a developer's
	  personal itch.

	\item Treating your users as co-developers is your least-hassle route
	  to rapid code improvement and effective debugging.

	\item Release early. Release often. And listen to your customers.

	\item Given a large enough beta-tester and co-developer base, almost
	  every problem will be characterized quickly and the fix obvious to
	  someone.

  \end{itemize}

\end{frame}

\begin{frame}
  \frametitle{Bazaar SW Development Rules}

  \begin{itemize}
	\item If you treat your beta-testers as if they're your most valuable
	  resource, they will respond by becoming your most valuable resource.

	\item To solve an interesting problem, start by finding a problem that
	  is interesting to you.

	\item Provided the development coordinator has a communications medium
	  at least as good as the Internet, and knows how to lead without
	  coercion, many heads are inevitably better than one.

  \end{itemize}

%  \begin{enumerate}
%	\item Good programmers know what to write. Great ones know what to
%	  rewrite (and reuse).
%
%	\item Plan to throw one away; you will, anyhow. (Copied from Frederick
%	  Brooks' The Mythical Man Month)
%
%	\item If you have the right attitude, interesting problems will find
%	  you.
%
%	\item When you lose interest in a program, your last duty to it is to
%	  hand it off to a competent successor.
%
%	\item Smart data structures and dumb code works a lot better than the
%	  other way around.
%
%	\item The next best thing to having good ideas is recognizing good
%%	  ideas from your users. Sometimes the latter is better.
%
%	\item Perfection (in design) is achieved not when there is nothing more
%	  to add, but rather when there is nothing more to take away.
%	  (Attributed to Antoine de Saint-Exupéry)
%
%	\item Any tool should be useful in the expected way, but a truly great
%	  tool lends itself to uses you never expected.
%
%	\item When writing gateway software of any kind, take pains to disturb
%	  the data stream as little as possible—and never throw away
%	  information unless the recipient forces you to!
%
%	\item A security system is only as secure as its secret. Beware of
%	  pseudo-secrets.
%
%  \end{enumerate}<++>

\end{frame}

\subsection{Examples}

\begin{frame}
  \frametitle{Enterprise}

  \begin{itemize}
	\item MS Windows
	\item OSX
	\item SAP
	\item Lotus
  \end{itemize}<++>

\end{frame}

\begin{frame}
  \frametitle{Applications}

  \begin{block}{Cathedral like}
	\begin{itemize}
	  \item Firefox
	  \item Chrome
	  \item OpenOffice
	\end{itemize}
  \end{block}

  \begin{block}{Bazaar like}
	\begin{itemize}
	  \item Linux Kernel
	  \item Pidgin
	  \item Exaile
	\end{itemize}
  \end{block}

\end{frame}

\begin{frame}
  \frametitle{Distributions}

  \begin{block}{Cathedral like}
	\begin{itemize}
	  \item RedHat
	  \item Fedora
	  \item Suse
	  \item Ubuntu
	\end{itemize}
  \end{block}

  \begin{block}{Bazaar like}
	\begin{itemize}
	  \item Archlinux
	  \item Debian
	\end{itemize}
  \end{block}

\end{frame}

\section{Getting into the Community}<++>

\subsection{How to}<++>

\begin{frame}
  \frametitle{Educate yourself}

  \begin{itemize}
	\item Know the tools used by the proyect
	  \begin{itemize}
		\item Make
		\item Autotools
		\item CMake
		\item Git, Bzr, Hg, Mtn
	  \end{itemize}
	\item Get to know the problem the software/module solves.
  \end{itemize}<++>

\end{frame}

\begin{frame}
  \frametitle{Get involved}

  \begin{itemize}
	\item Join the communication channels
	\item Join the Mailing Lists.
	\item Join relevant IRC channels.
  \end{itemize}<++>

\end{frame}

\begin{frame}
  \frametitle{Mailing Lists}

  \begin{itemize}
	\item The most common offline communication for worldwide.
  \end{itemize}<++>

\end{frame}

\begin{frame}
  \frametitle{IRC}

  \begin{itemize}
	\item Effective online conference.
	\item Conference likEffective online conference.
	\item Conference like.
  \end{itemize}

\end{frame}

\section{Low Lights in FLOSS Community}

\begin{frame}
  \frametitle{FLOSS also have lowligths}

  \begin{itemize}
	\item Ego clashes.
	\item Harrassment.
	\item Ada Initiative. (not quite sure about this one).
  \end{itemize}<++>

\end{frame}

\section{Nettiquete}<++>

\begin{frame}
  \frametitle{<+Title+>}

  \begin{block}{<+Title+>} % alertblock, exampleblock
    <++>
  \end{block}

  %\alert{<+Alert+>}
\end{frame}

<++>

\end{document}


